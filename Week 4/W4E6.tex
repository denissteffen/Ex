\documentclass[a4paper,11pt,french]{article}

    \usepackage[utf8]{inputenc}
    
    \usepackage{mathrsfs}
    \usepackage[english]{babel}
    \usepackage{mathtools} % includes amsmath
    \usepackage{amssymb}
    \usepackage{amsthm}
    \usepackage{amscd}
    \usepackage{todonotes}
    
    \usepackage{multirow}
    \usepackage{enumerate}
    
    \usepackage{tikz}
    \usepackage{framed}
    \usepackage[colorlinks]{hyperref}
    
    
    \title{Discrete Optimization: Homework \#4, Ex. \#6}
    \author{Denis Steffen, Yann Eberhard \& Gaëtan Bossy}
    
    \begin{document}
    
    \maketitle
    Let $X=\{ a_{1}, ... , a_{t}\}$ be a finite set of $\mathbb{R}^{n}$, we will prove that cone$(X)$ is closed and convex.
    \\
    \\
    By Caratheodory's theorem, $\forall x \in$ cone$(X)$ there exists $\widetilde{X}=\{ a'_{1}, ... , a'_{k}\} \subseteq X$ such that $x \in \widetilde{X}$ and the vectors $\{ a'_{i}\}$ in $\widetilde{X}$ are linearly independent. 
    Let $A$ be the matrix with $a_{i}^{\prime T}$ ($i=1, ... , k$) on the k-first rows and for the last $n-k$ rows of $A$ can be obtained by the Gram-Schmidt process from the first $k$ rows. $A$ is invertible since it is a $n \times n$ matrix and all the rows are linearly independent.
    \\
    By exercise \#2, cone$(\widetilde{X})=\{ x\in \mathbb{R}^{n} : a_{ i}^{\prime -1}x \geq 0, i=1, ... , k ; a_{j}^{\prime -1}x = 0, j=k+1, ... , n  \}$.
    Since cone$(\widetilde{X})$ is the finite intersection of the pre-image by a linear application of a closed set (either $[0, \infty[$ or $\{0\}$), it is also closed. 
    \\
    There is a finite number of $\widetilde{X}$ since $X$ is finite. So, we can enumerate all such $\widetilde{X}$. We will prove that cone$(X)=\displaystyle{\bigcup_{ i=1}^{m}}$ cone$(\widetilde{X_{i}})$. 
    \\
    By construction, $\forall x \in$ cone$(X)$ $\exists j \in \{1, ..., m\}$ such that $x \in$ cone$(\widetilde{X_{j}})$. On the other way, if $x \in \displaystyle{\bigcup_{ i=1}^{m}}$ cone$(\widetilde{X_{i}})$, there exists $j \in \{1, ... , m \}$ such that $x \in$ cone$(\widetilde{X_{j}})$. 
    By definition of a cone-hull and since every vectors of $\widetilde{X}$ are also in cone$(X)$ ($\widetilde{X_{j}} \subseteq X$), this implies $x \in$ cone$(X)$.
    \\
    That's why, cone$(X)$ is closed because it is equal to a finite union of closed sets.
    \\
    \\
    $\forall x, y \in$ cone$(X)$, $\forall \lambda \in [0,1]$ we need to show $\lambda 
    x +(1-\lambda)y \in$ cone$(X)$. $x=\displaystyle{\sum_{i=1}^{t}} \alpha_{i} a_{i}$ and $y=\displaystyle{\sum_{i=1}^{t}} \beta_{i} a_{i}$ since $x,y \in$ cone$(X)$.
    \\
    $\lambda x +(1-\lambda)y = \displaystyle{\sum_{i=1}^{t}} (\lambda \alpha_{i} + (1-\lambda)\beta_{i}) a_{i}$, we can replace $\lambda \alpha_{i} + (1-\lambda)\beta_{i}$ by $\gamma_{i}$. Since $\lambda \in [0,1]$ and $\alpha_{i}, \beta_{i} \in \mathbb{R}_{\geq 0}$, $\gamma_{i} \geq 0$. This implies that cone$(X)$ is convex :
    \begin{equation*}
    \lambda x +(1-\lambda)y = \displaystyle{\sum_{i=1}^{t}} \gamma_{i} a_{i} \in cone(X)
    \end{equation*}
    
    
    \end{document}
