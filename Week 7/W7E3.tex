\documentclass[a4paper,11pt,french]{article}

    \usepackage[utf8]{inputenc}
    
    \usepackage{mathrsfs}
    \usepackage[english]{babel}
    \usepackage{mathtools} % includes amsmath
    \usepackage{amssymb}
    \usepackage{amsthm}
    \usepackage{amscd}
    \usepackage{todonotes}
    
    \usepackage{multirow}
    \usepackage{enumerate}
    
    \usepackage{tikz}
    \usepackage{framed}
    \usepackage[colorlinks]{hyperref}
    \usepackage[T1]{fontenc}
    
  
    
    \title{Discrete Optimization: Homework \#7, Ex. \#3}
    \author{Denis Steffen, Yann Eberhard \& Gaëtan Bossy}
    
    \begin{document}
    
    \maketitle
   We consider the following LP :  max$\{ c^Tx : Ax \leq b, b\in \mathbb{R}^n \} $ and assume that it is feasible and bounded. Thus, we know that there exists an optimal solution, say $x^*$. 
   We also know that the dual of our LP : min$ \{ b^Ty : A^Ty = c, y \geq 0 \} $ is feasible and bounded by the Strong Duality theorem. That's why, we have that, for $y^*$ an optimal solution of the dual LP : 
   $c^Tx^* = b^Ty^*$ i.e. the optimal values coincide. 

   But $x^*$ is optimal for the primal LP and $y^*$ for the dual LP. So for all $(x^*, y^*)$ satisfying the previous equation and the respective constraints of primal and dual LP, $(x^*, y^*)$ will be optimal. 
   In other words, this means that all $(x, y)$ in the following Polyhedron P are optimal.
\begin{displaymath}
     c^Tx = b^Ty
    \end{displaymath} 
    \begin{displaymath}
       Ax \leq b
     \end{displaymath}
     \begin{displaymath}
      A^Ty = c
    \end{displaymath}
    \begin{displaymath}
      y \geq 0
    \end{displaymath}
    We have $ P = \{ \tilde{x} \in \mathbb{R}^n : \tilde{A} \tilde{x} \leq \tilde{b} \} $ with $\tilde{A} =  \begin{pmatrix}
       c^T & -b^T\\
      -c^T & b^T\\
      A & 0\\
      0 & A^T\\
      0 & -A^T\\
      0 & -I\\
    \end{pmatrix},
      \tilde{x} = 
    \left(\begin{array}{c}
      x\\
      y\\
    \end{array} \right)$ and $ \tilde{b} = \left( \begin{array}{c} 
    0\\
    0\\
    b\\
    c\\
    -c\\ 
    0\\
    \end{array} 
    \right).$ 


    \noindent Since we know that there exists the optimal solution $(x^*, y^*)$ of the LP, the vector $w = \left(\begin{array}{c}
    x^*\\
    y^*\\  
    \end{array}
   \right )$ is in P. So using the oracle algorithm on the Polyhedron P in a single call, we find an optimal solution of our LP.
  
  \end{document}
