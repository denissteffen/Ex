\documentclass[a4paper,11pt,french]{article}

    \usepackage[utf8]{inputenc}
    
    \usepackage{mathrsfs}
    \usepackage[english]{babel}
    \usepackage{mathtools} % includes amsmath
    \usepackage{amssymb}
    \usepackage{amsthm}
    \usepackage{amscd}
    \usepackage{todonotes}
    
    \usepackage{multirow}
    \usepackage{enumerate}
    
    \usepackage{tikz}
    \usepackage{framed}
    \usepackage[colorlinks]{hyperref}
    
    
    \title{Discrete Optimization: Homework \#5, Ex. \#5}
    \author{Denis Steffen, Yann Eberhard \& Gaëtan Bossy}
    
    \begin{document}
    
    \maketitle
    
    We have to prove that $P = \{ x \in \mathbb{R}^n : Ax \leq b\}$, $P \not = \emptyset $, contains a line if and only if $A$ does not have full column-rank.
    We know from linear algebra that $A$ does not have full column-rank if ker$(A) \not = \mathbf {0}$. 
\\

    First, we prove  $\Rightarrow$ : \\
    By hypothesis, we know that P contains a line, i.e. there exits a nonzero $v \in \mathbb{R}^n$ and a $x^* \in \mathbb{R}^n$ such that for all $\lambda \in \mathbb{R}$ we have $ p \coloneqq x^* + \lambda \cdot v \in P$. 
    We define the rows of $A \in \mathbb{R}^{m \times n}$ by $a_j$.
    We need to prove that there exists a nonzero vector $u \in$ ker$(A)$. We choose $u = v$ and let's prove that $Av = \mathbf 0$.  
    We'll prove it by contradiction : we suppose that there exists an index $1 \leq i \leq m$ such that $a_i^T*v \not = 0$. We have that, for all $\lambda$ : 
    \begin{equation}
        a_i^Tp = a_i^Tx^* + \lambda \cdot a_i^Tv
    \end{equation}
    If we take  $\lambda = \frac{(b_i + \epsilon) - a_i^T*x^*}{a_i^T*v} $ ($\epsilon > 0$), (1) gives $a_i^Tp = b_i + \epsilon > b_i$. 
    This means that $p \not \in P$ but this is a contradiction. Thus, $A$ does not have full column-rank.
    \\ 

    Then, we prove $\Leftarrow$ : 
    \\ Suppose there exists a nonzero vector $v \in \mathbb{R}^n$ such that $Ax = 0$. 
    Let $x^* \in P$, so $Ax^* \leq b$. We consider the vector $p_\lambda \coloneqq x^* + \lambda \cdot v$, for $\lambda \in \mathbb{R}$. 
    We get : 
    \begin{equation*}
        Ap_\lambda = Ax^* + \lambda \cdot Av = Ax^* \leq b
    \end{equation*}
    Thus, $p_\lambda \in P$ and this is verified for all $\lambda$. This implies that $P$ contains a line.

    
    \end{document}
