\documentclass[a4paper,11pt,french]{article}

    \usepackage[utf8]{inputenc}
    
    \usepackage{mathrsfs}
    \usepackage[english]{babel}
    \usepackage{mathtools} % includes amsmath
    \usepackage{amssymb}
    \usepackage{amsthm}
    \usepackage{amscd}
    \usepackage{todonotes}
    
    \usepackage{multirow}
    \usepackage{enumerate}
    
    \usepackage{tikz}
    \usepackage{framed}
    \usepackage[colorlinks]{hyperref}
    \usepackage[T1]{fontenc}
    
  
    
    \title{Discrete Optimization: Homework \#8, Ex. \#1}
    \author{Denis Steffen, Yann Eberhard \& Gaëtan Bossy}
    
    \begin{document}
    
    \maketitle
We use basic binary search on each of the dimension. For every $1\leq i\leq n$, we use the oracle algorithm at most $log_2(B)$ times and find the i-th coordinate of the red point. By doing this $n$ times, we can find every coordinate of the unique red point. Thus our main algorithm will call the secondary algorithm at most $n$ times, and the secondary algorithm will call the oracle algorithm %$log_2(B+1)=$
$O(log(b))$ times, which means the main algorithm will call the oracle algorithm $O(n\cdot log(B))$ times.\\ \\
Main algorithm: $main(B)$, with $B\in\mathbb{N}$\\
For $i=1\rightarrow n$:
	 \begin{verse}
	 $x^*_i=alg(i,0,B)$
	 \end{verse}
return $x^*$\\ \\
Secondary algorithm: $alg(i,a,b)$, with $i$ an index, $a,b\in\mathbb{N}$\\
If $a=b$
\begin{verse}
return $a$
\end{verse}
else
\begin{verse}
 if $oracle(i,a+floor(b-a/2))=true$ 
 \begin{verse}
 return $alg(i,a,a+floor(b-a/2))$ 
 \end{verse}
 else
 \begin{verse} return $alg(i,a+floor(b-a/2),b)$
 \end{verse}
\end{verse}
  \end{document}
