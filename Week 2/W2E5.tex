\documentclass[a4paper,11pt,french]{article}

\usepackage[utf8]{inputenc}

\usepackage{mathrsfs}
\usepackage[english]{babel}
\usepackage{mathtools} % includes amsmath
\usepackage{amssymb}
\usepackage{amsthm}
\usepackage{amscd}
\usepackage{todonotes}

\usepackage{multirow}
\usepackage{enumerate}

\usepackage{tikz}
\usepackage{framed}
\usepackage[colorlinks]{hyperref}


\title{Discrete Optimisation: Homework \#2, Ex. \#8}
\author{Denis Steffen, Yann Eberhard \& Gaëtan Bossy}

\begin{document}

\maketitle
We know that $\forall A,B\in\mathbb{R}^{n\cdot n},\,\,det(A\cdot B)=det(A)\cdot det(B)$. So if we prove that $det((M_{2^k})^2)=(2^k)^{2^k}$, it'll mean $|det(M_{2^k})|=(2^k)^{2^{k-1}}$. One can easily compute this multiplication with blocks: 
\begin{equation*}
M_{2^k}\cdot M_{2^k}=
\begin{pmatrix}
    2\cdot(M_{2^{k-1}})^2 & M_{2^{k-1}}-M_{2^{k-1}} \\ 
    M_{2^{k-1}}-M_{2^{k-1}} &  2\cdot(M_{2^{k-1}})^2
  \end{pmatrix}=
  \begin{pmatrix}
    2\cdot(M_{2^{k-1}})^2 & 0_{2^{k-1}\cdot 2^{k-1}} \\ 
    0_{2^{k-1}\cdot 2^{k-1}} &  2\cdot(M_{2^{k-1}})^2
  \end{pmatrix}
 \end{equation*}
 This means that if $(M_{2^{k-1}})^2$ is diagonal, then $(M_{2^{k}})^2$ is also a diagonal matrix. As $(M_{2^{0}})^2=(1)$ is a diagonal matrix, this means that $(M_{2^{k}})^2$ is diagonal for every $k\in\mathbb{N}$. The determinant of a diagonal matrix being the product of its diagonal elements, we just need to compute these. In our previous calculation, we saw that the diagonal elements on $M_{2^k}^2$ were twice as large as the diagonal elements of $M_{2^{k-1}}^2$. As the diagonal element of $M_{2^0}^2$ is 1, we can easily deduce that the diagonal elements on $M_{2^k}^2$ are $2^k$. This means that $det(M_{2^k}^2)=(2^k)^{2^k}$, which proves that $|det(M_n)|=n^{\frac{n}{2}}$.
  

\end{document}